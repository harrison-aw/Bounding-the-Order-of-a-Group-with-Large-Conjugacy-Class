\documentclass[main.tex]{subfiles}

\begin{document}

\chapter{An Analog to Durfee's Relative Parameter}

In this section, we consider a generalization of the parameter $e$ that incorporates a normal subgroup. For $x \in G$ and $N \normal G$, let $c_N(x) = |C_G(x):C_N(x)|$ and $k_N(x) = |G:NC_G(x)|$. We then define $e_N$, the relative parameter, as follows:
$$e_N = min\{(c_N(x) - 1) \cdot \sqrt{k_N(x)} : x \in G\}\text{.}$$
If $N=1$, then $c_N(x) = |C_G(x)|$ and $k_N(x) = |G:C_G(x)|$ which yields $e_N = min\{(|C_G(x)| - 1)\cdot\sqrt{|G:C_G(x)|}\} = e$. This new parameter thus generalizes our previous analog. Note that $e_N$ is distinct from our parameter $e$ for the factor group. An important difference is that $e_N$ can be zero for non-trivial groups (see Theorem \ref{en0}). Therefore, there are groups $G$ for which $G/N$ has a non-zero $e$ but $G$ has $e_N = 0$ for the same normal subgroup (in fact, this is true for all pairs $(G, N)$ such that $e_N = 0$).



 Before we analyze $e_N$, we introduce a definition used by J. Britnell and M. Wildon \cite{britnellwildonarticle}.

\hss

\begin{definition}
Let $G$ be a finite group and $N \normal G$. A conjugacy class $x^G$ of $G$ is \emph{non-split} if $x^G = x^N$.
\end{definition}

\hss

The following lemma establishes a useful result regarding the behavior of $c_N(x)$ and $k_N(x)$.

\begin{lemma}\label{kn1}
Let $G$ be a finite group with normal subgroup $N$. For $x \in G$, $c_N(x)$ and $k_N(x)$ are positive integers. The conjugacy class of $x$ is non-split if and only if $k_N(x) = 1$.
\end{lemma}

\begin{proof}
The first part of the lemma follows directly from the fact that both $c_N(x)$ and $k_N(x)$ are indices of finite groups.

For the next part, consider $|NC_G(x)|$. It satisfies the following identity:
$$|NC_G(x)| = \frac{|N| \cdot |C_G(x)|}{|C_N(x)|}\text{.}$$

Now
\begin{align*}
k_N(x) &= |G : NC_G(x)| \\
&= |G| \cdot \frac{|C_N(x)|}{|N| \cdot |C_G(x)|} \\
&= \frac{|G : C_G(x)|}{|N : C_N(x)|}
\end{align*}
and thus $k_N(x) = 1$ if and only if $|x^N| = |x^G|$. But, since $x^N \subseteq x^G$, it must be the case that $x^N = x^G$ and thus $x^G$ is non-split.
\end{proof}

We begin by noting that $e_N \ge 0$ since $c_N(x)$ and $k_N(x)$ are both greater than 1. This brings us to our first result:

\begin{theorem}\label{en0}
Let $G$ be a finite group with normal subgroup $N$. The parameter $e_N = 0$ if and only if $C_G(x) \le N$ for some $x \in G$. In particular, $e_N > 0$ if $N$ contains no centralizers.
\end{theorem}

\begin{proof}
Suppose $e_N = 0$. Then there exists an $x \in G$ such that $(c_N(x) - 1) \cdot \sqrt{k_N(x)} = 0$. Since $k_N(x) \ge 1$, we must have $c_N(x) = 1$. Then $C_G(x) = C_N(x)$ and $C_G(x) \le N$. Reversing this chain of logic gives the converse.
\end{proof}

Zero is the minimum value $e_N$ can take since both $c_N(x)$ and $k_N(x)$ are greater than or equal to 1 and $e_N$ is non-decreasing with respect to both $c_N(x)$ and $k_N(x)$ (see the previous chapter for a discussion of this; the argument is the same). The next smallest value of $e_N$ is 1. For $e_N > 0$, we must have $c_N(x) \ge 2$. Since $e_N$ is non-decreasing with respect to $k_N(x)$, we may conclude that $e_N = 1$ is the smallest non-zero value $e_N$ can take. This mirrors the situation with the non-relative $e$ in that the values $e_N$ are restricted to only a countably infinite subset of the reals (the values of $e_N$ are indexed by pairs of integers).

There are many pairs $(G, N)$ with $e_N = 0$. Since the parameter is defined as a minimum, if $N$ contains any centralizers, it must fall into this class of groups. Notably this includes Frobenius groups which were ubiquitous in the last chapter. This gives further evidence that the relative $e_N$ is distinct from the non-relative $e$.

The next theorem continues our classification of specific values of $e_N$ and considers the case when $e_N = 1$.

\begin{theorem}\label{en1}
The parameter $e_N = 1$ if and only if $|G:N| = 2$ and $x^G$ is non-split.
\end{theorem}

\begin{proof}
Suppose $e_N = 1$. Then there exists an $x \in G$ such that $(c_N(x) - 1) \cdot \sqrt{k_N(x)} = 1$. Because $c_N(x) - 1$ is always an integer and $k_N(x) \ge 1$, it must follow that $c_N(x) = 2$ and $k_N(x) = 1$. By Lemma \ref{kn1} then, $x^G$ is non-split. Also, we have $|G : N| = |C_G(x) : C_N(x)| = 2$ since
\begin{align*}
1 = k_N(x) &= |G : NC_G(x)| \\
&= \frac{|G : N|}{|C_G(x) : C_N(x)|} \text{.}
\end{align*}

Now, let $|G:N| = 2$ and $x^G = x^N$. Then $k_N(x) = 1$ by Lemma \ref{kn1}. We may infer then that $2 = |G : N| = |C_G(x) : C_N(x)| = c_N(x)$. Thus, $e_N = 1$. 
\end{proof}

The easiest examples of pairs with $e_N = 1$ are even order abelian groups paired with a normal subgroup of index 2. Suppose $G$ is an even-order abelian group. Then $G$ contains an involution $t$ and $G = N \times \generate{t}$. Since $G$ is abelian and $N$ is proper, $N$ cannot contain any centralizers. Furthermore, a conjugacy class containing a single element cannot be split. Thus, $e_N = 1$ for pairs $(G, N)$ with $G$ abelian and $|G : N| = 2$. A non-abelian example is not known.

The next value of interest is $\sqrt{2}$. It seems potentially possible for $e_N = \sqrt{2}$. The reasoning here is similar to that used for the non-relative $e$. However, an example that gives this particular value is, at present, unknown. It may be the case that the relative $e_N$ is very restrictive on the relationship between $G$ and a normal subgroup $N$ and that pairs where $e_N > 1$ are comparatively rare. With that in mind, we move on to our next theorem.

\begin{theorem}
If $e_N = p$ for a prime $p$, then either:
\begin{enumerate}
	\item $|G:N| = p + 1$ and some conjugacy class $x^G$ of $G$ is non-split.
	\item $|G:N| = 2p^2$ and some conjugacy class $x^G$ of $G$ is split.
\end{enumerate}
\end{theorem}

\begin{proof}
Let $e_N = p$. There exists an $x \in G$ such that $(c_N(x) - 1) \cdot \sqrt{k_N(x)} = p$. Since $c_N(x) - 1$ is an integer and $k_N(x) \ge 1$, both $c_N(x) - 1$ and $\sqrt{k_N(x)}$ must be factors of $p$ (i.e., integers whose product is $p$). Then we may infer, because $p$ is prime, that either:
\begin{enumerate}
	\item $c_N(x) - 1 = p$ and $\sqrt{k_N(x)} = 1$ or
	\item $c_N(x) - 1 = 1$ and $\sqrt{k_N(x)} = p$.
\end{enumerate}

\subsubsection*{Case 1} Suppose $c_N(x) = p + 1$ and $k_N(x) = 1$. By Lemma \ref{kn1}, $x^G = x^N$. All that remains to be shown is the index of $N$ in $G$. We have
$$1 = k_N(x) = \frac{|G : N|}{|C_G(x):C_N(x)|} = \frac{|G : N|}{p + 1}\text{.}$$
Then $|G:N| = p+1$.

\subsubsection*{Case 2} Suppose $c_N(x) = 2$ and $k_N(x) = p^2$. Consider $k_N(x)$:
$$k_N(x) = \frac{|G:N|}{c_N(x)}\text{.}$$
Then $|G:N| = k_N(x) \cdot c_N(x) = 2p^2$. Also, since $k_N(x) > 1$, $x^G \ne x^N$ and $x^G$ is split.
\end{proof}

The former case is satisfied by certain abelian groups. Let $G$ be abelian and $x \in G$ have order $p + 1$ for a prime $p$. Then $G = N \times \generate{x}$ for some normal subgroup $N$ and so $e_N = p$ for $(G, N)$. As above, a non-abelian example is not known at present. Such an example would require a non-abelian group with a normal subgroup containing no centralizers, a condition that warrants further study.

\end{document}