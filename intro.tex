\documentclass[main.tex]{subfiles}

\begin{document}

\chapter{Introduction}

This thesis considers a problem in a branch of abstract algebra known as finite group theory. Finite group theory is concerned with the structure of finite sets of objects related to one another by a binary operator such as mutliplication or addition. Often, these structures represent various forms of symmetry. In this thesis, we study a problem inspired by a series of publications.

\hss 

In 2008, Noah Snyder \cite{snyderarticle} published a result on a group parameter $e$. He classified groups based on fixed values of $e$ and bounded the order of a group in terms of $e$. His definition was this:

\hss

\begin{definition}
Let $G$ be a group of order $n$ and $V$ an irreducible representation of $G$ over $\mathbb{C}$ of dimension $d$. Define $e$ to be the non-negative integer satisfying $n = d(d + e)$.
\end{definition}

\hss

The parameter $e$ must be an integer since $d$ divides $|G|$ \cite[p.96]{gorensteinfinitegroups}. It has the interesting property that if $e$ is small relative to $d$, then $G$ has a character of large degree.

Before Snyder, Yakov Berkovich classified groups where $e=1$ and $e=2$ \cite{berkovicharticle}. Snyder added to this classification by solving the case where $e=3$. His most noteworthy contribution to this problem, though, is the following theorem:

\hss

\begin{theorem}[Snyder \cite{snyderarticle}]
Let $G$ be a group of order $n$ and $d$ be the degree of some irreducible character of $G$ and $d(d+e) = n$. If $e > 1$, $n \le ((2e)!)^2$.
\end{theorem}

\hss

This sparked a series of publications aiming to improve upon the bound. The first article along this line of inquiry was written by Isaacs \cite{isaacsarticle}. The relevant theorem from his article is as follows:

\hss

\begin{theorem}[Isaacs \cite{isaacsarticle}] Let $G$ be a group of order $n$ and $d$ be the degree of some irreducible character of $G$. If $e > 1$, then $n \le Be^6$ for some universal constant $B$.
\end{theorem}

\hss

This result required the simple group classification theorem for full generality, but succeeded in giving a polynomial bound. Interestingly, it spawned another paper \cite{larsenmalletieparticle} whose results were necessary to complete certain cases of Isaacs' proof.

Two students of Isaacs, Christina Durfee and Sara Jensen, were the next to make headway on the problem. They removed the universal constant from Isaacs' bound. Their main result was the following:

\hss

\begin{theorem}[Durfee-Jensen \cite{durfeejensenarticle}]
For $e > 1$, we have the follwing bounds on $|G|$ in terms of $e$:
\begin{enumerate}
	\item If $e$ is divisible by two distinct primes, then $|G| < e^4 + e^3$.
	\item If $e$ is a prime power then $|G| < e^6 - e^4$.
	\item If $e$ is a prime, then $|G| < e^4 + e^3$.
\end{enumerate}
\end{theorem}

\hss

\noindent Since $e$ is an integer, this theorem has as an immediate corollary that $|G| < e^6 - e^4$ for all groups $G$. It is still a polynomial of degree 6, but is significantly smaller than Isaacs' bound.

Mark Lewis made the most recent advance. He gave general conditions \cite{lewisarticle} for a group to satisfy the stronger bound $|G| \le e^4 - e^3$.

\hss

\begin{theorem}[Lewis \cite{lewisarticle}] Let $G$ be a group with a nontrivial, abelian normal subgroup. Let $d$ be the degree of some irreducible character of $G$ and $|G| = d(d+e)$. If $e > 1$, then $d \le e^2 - e$ and $|G| \le e^4 - e^3$. This bound is best possible.
\end{theorem}

\hss

\noindent Notably, all solvable groups satisfy the conditions of this theorem. Since there exist solvable groups where $|G| = e^4 - e^3$ (a result due to Isaacs \cite{isaacsarticle}), this problem is completely solved for solvable groups. 

\hss

A related problem was the subject of Durfee's dissertation \cite{durfeedissertation}. She studied a parameter that is essentially Snyder's $e$ relative to a fixed normal subgroup. 

\hss

\begin{definition}
Let $N$ be a normal subgroup of a finite group $G$. Let $\chi$ and $\Theta$ be irreducible characters of $G$ and $N$, respectively, such that $\Theta$ is fixed by the conjugation action of $G$ and $\chi$ restricts to a multiple of $\Theta$ on $N$. Let $d = \chi(1)/\Theta(1)$. Define $e$ by $|G/N| = d(d+e)$.
\end{definition}

\hss

\noindent Durfee's parameter, like Snyder's, is always a non-negative integer. In her dissertation, she studies the case where $e = 1$ and $e = 2$ and gives consideration to supersolvable groups and nilpotent groups. Her main result, though, was the following:

\hss

\begin{theorem}[Durfee \cite{durfeedissertation}]
Let $N$ be a normal subgroup of a finite group $G$ where $G/N$ is solvable and let $\Theta$ be an irreducible character of $N$ that is $G$-invariant. Let $\chi$ be an irreducible character of $G$ that is a multiple of $\Theta$ and let $d = \chi(1)/\Theta(1)$. Write $|G:N| = d(d+e)$ for some non-negative integer $e$. If $e \ge 1$ and $d > e^5 - e$, then we can find groups $X$ and $Y$ such that:
\begin{enumerate}
	\item $N \subseteq X \normal Y \subseteq G$
	\item $|Y/X| = (d/e)(d/e+1)$
	\item $Y/X$ is either the group of order 2 or is a 2-transitive Frobenius group.
\end{enumerate}
\end{theorem}

\hss

In light of the aforementioned research, a group theoretic analog is studied in this thesis. There is a close connection between characters and conjugacy classes that often motivates a problem for one based off a result of the other. Sometimes, the theorems produced from this line of research are very similiar. A survey of such results can be found in \cite{chillagarticle}.

Our approach is to preserve the following identity:
\begin{equation}\label{degreeidentity}
|G| = \sum_{i=1}^n d_i^2
\end{equation}
where $d_i$ is the degree of an irreducible character and $i$ indexes the $n$ ordinary irreducible characters of $G$. This is a well-known identity that holds for all finite groups \cite{gorensteinfinitegroups}. We thus want to replace the $d_i$'s with analogous values for conjugacy classes. Since the number of irreducible characters is the same as the number of conjugacy classes \cite[p.96]{gorensteinfinitegroups}, this is a reasonable goal. Furthermore, the sizes of conjugacy classes satisfy a similar identity:
$$|G| = \sum_{i=1}^n |x_i^G|$$
where there are $n$ distinct conjugacy classes and the $x_i \in G$ are representatives of each of these conjugacy classes. The primary difference between the two identities is the presence of squared terms in the former. The simplest way to accomodate this is to consider the square root of conjugacy class sizes instead of just conjugacy class sizes:
$$|G| = \sum_{i=1}^n \left(\sqrt{|x_i^G|}\right)^2\text{.}$$
This maintains the form of (\ref{degreeidentity}) even though our terms are no longer rational. The main benefit of this change is that it makes the definition of an analogous $e$ resemble the original very closely.

Taking $d$ to be the square root of a conjugacy class, we can define an analog $e$ with Snyder's equation: $|G| = d(d+e)$. We still have that  $d^2 + de$ is a sum of non-negative integers but, unfortunately, that is the extent of the parallel. This analog can take on irrational values. It does, however, have the property that small values of $e$ relative to $d$ correspond to large conjugacy class sizes (see Chapter 5). For non-trivial groups, this indicates that our $e$ is an adequate analog. In this thesis, we bound the group order in terms of this analogous $e$ and then classify groups attaining the bound.

We also consider a relative parameter in the spirit of Durfee's research. Let $N$ be a fixed normal subgroup of $G$. Let $x \in G$ and $d = \sqrt{|x^G|/|x^N|}$. Define $e$ via $|G/N| = d(d+e)$. Here we have the same issue as above: $e$ can be irrational. In spite of this, there are still some worthwhile results for the analog. We will provide results for fixed values of $e$ and discuss the groups associated with them.

The rest of this paper proceeds with chapters of background material and then two chapters for our main results, one for Snyder's analog and the other for Durfee's analog. A solid understanding of group theory is assumed in the following although, where possible, we try to present all supporting results. When it is not feasible to provide a proof, a reference is given.

Our notation will follow closely that used in Isaacs' \emph{Finite Group Theory} \cite{isaacsfinitegrouptheory} and other notation will be introduced as needed.
\end{document}