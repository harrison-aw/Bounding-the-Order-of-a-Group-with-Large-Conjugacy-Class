\documentclass[main.tex]{subfiles}

\begin{document}

\chapter{Introduction}

In 2008, Noah Snyder published \cite{snyderarticle} a result on a group parameter $e$. He classified groups based on fixed values of $e$ and bounded the order of a group with it. His definition is this:

\hss

\begin{definition}
Let $G$ be a group of order $n$ and $V$ an irreducible representation over $\mathbb{C}$ of dimension $d$. Define $e$ to be the non-negative integer $n = d(d + e)$.
\end{definition}

\hss

The parameter $e$ must be an integer since $d$ divides $|G|$. It has the interesting property that if $e$ is small relative to $d$, then $G$ has a character of large degree.

Before Snyder, Yakov Berkovich classified groups where $e=1$ and $e=2$ \cite{berkovicharticle}. Snyder added to this classification by solving the case where $e=3$. His most noteworthy contribution to this problem, though, is the following theorem:

\hss

\begin{theorem}[Snyder \cite{snyderarticle}]
Let $G$ be a group of order $n$ with a simple $\C[G]$-module $V$ of dimension $d$ and $d(d+e) = n$. If $e > 1$, $n \le ((2e)!)^2$.
\end{theorem}

\hss

This sparked a series of publications aiming to improve upon the bound. The first article along this line of inquiry was written by Isaacs \cite{isaacsarticle}. The relevant theorem from his article:

\hss

\begin{theorem}[Isaacs \cite{isaacsarticle}] Let $G$ be a group of order $n$ and $d$ be the degree of some irreducible character of $G$. If $e > 1$, then $n \le Be^6$ for some universal constant $B$.
\end{theorem}

\hss

This result required the simple group classification theorem for full generality, but succeeded in giving a polynomial bound. Interestingly, it spawned another paper \cite{larsenmalletieparticle} whose results were necessary to complete certain cases of Isaacs' proof.

Two students of Isaacs, Christina Durfee and Sara Jensen, were the next to make headway on the problem. They removed the universal constant from Isaacs bound and then some. Their main result was the following:

\hss

\begin{theorem}[Durfee-Jensen \cite{durfeejensenarticle}]
For $e > 1$, we have the follwing bounds on $|G|$ in terms of $e$:
\begin{enumerate}
	\item If $e$ is divisible by two distinct primes, then $|G| < e^4 + e^3$.
	\item If $e$ is a prime power then $|G| < e^6 - e^4$.
	\item If $e$ is a prime, then $|G| < e^4 + e^3$.
\end{enumerate}
\end{theorem}

\hss

\noindent Since $e$ is an integer, this theorem has as an immediate corollary that $|G| < e^6 - e^4$ for all groups $G$. It is still a polynomial of degree 6, but is significantly smaller than Isaac's bound.

Mark Lewis made the most recent advance. He gave general conditions \cite{lewisarticle} for a group to satisfy the stronger bound $|G| \le e^4 - e^3$ and showed that certain groups satisfy this bound, proving it to be best possible.

\hss

\begin{theorem}[Lewis \cite{lewisarticle}] Let $G$ be a group with a nontrivial, abelian normal subgroup. Let $d$ be the degree of some irreducible character of $G$ and $|G| = d(d+e)$. If $e > 1$, then $d \le e^2 - e$ and $|G| \le e^4 - e^3$. This bound is best possible.
\end{theorem}

\hss

\noindent Notably, all solvable groups satisfy the conditions of this theorem making it applicable to a large class of interesting groups.

A related problem was the subject of Durfee's dissertation \cite{durfeedissertation}. She studied a parameter that is essentially Snyder's $e$ relative to a fixed normal subgroup.

\hss

\begin{definition}
Let $N$ be a normal subgroup of a finite group $G$. Let $\chi$ and $\Theta$ be irreducible characters of $G$ and $N$, respectively, such that $\Theta$ is fixed by the conjugation action of $G$ and $\chi$ restricts to a multiple of $\Theta$ on $N$. Let $d = \chi(1)/\Theta(1)$. Define $e$ by $|G/N| = d(d+e)$.
\end{definition}

\hss

\noindent Durfee's parameter, like Snyder's, is always a non-negative integer. In her dissertation, she studies the case where $e = 1$ and $e = 2$ and gives consideration to supersolvable groups and nilpotent groups.

\hss

In light of the aforementioned research, a group theoretic analog is studied in this thesis. There is a close connection between characters and conjugacy classes that often motivates a problem for one based off a result of the other. Sometimes, the theorems produced from this line of research are very similiar. A survey of such results can be found in \cite{chillagarticle}.

Our approach is to preserve the following identity:
\begin{equation}\label{degreeidentity}
|G| = \sum_{i=1}^n d_i^2
\end{equation}
where $d_i$ is the degree of an irreducible character and $i$ indexes the $n$ character degrees of $G$. We want to replace the $d_i$'s with analogous values for conjugacy classes. Sizes of conjugacy classes satisfy a similar identity:
$$|G| = \sum_{i=1}^n |x_i^G|$$
where $|x_i^G|$ is a conjugacy class size and $i$ indexes $n$ distinct conjugacy class representitatives. Thus, we may take $d_i = \sqrt{|x_i^G|}$ to preserve (\ref{degreeidentity}).

Our parameter is then defined with the same equation as the others: $|G| = d(d+e)$ where $d$ is the square root of a conjugacy class size. Unfortunately, this new parameter does not have the nice property of being an integer but is non-negative. It also has the property that small values of $e$ relative to $d$ correspond to large conjugacy class sizes. This indicates that our $e$ is an adequate analog. In this thesis, we bound the group order in terms of this analogous $e$ and then classify groups satisfying the bound.

We also consider a relative parameter in the spirit of Durfee's research. Let $N$ be a fixed normal subgroup of $G$. Let $x \in G$ and $d = \sqrt{|x^G|/|x^N|}$. Define $e$ via $|G/N| = d(d+e)$. Here we have the same issue as before that $e$ is not an integer. In spite of this, there are still some worthwhile results for the analog. We will provide a general bound and give results for fixed values of $e$.

The rest of this paper proceeds with chapters of background material grouped by topics and then two chapters for our main results, one for Snyder's analog and the other for Durfee's analog. A solid understanding of group theory is assumed in the following though, where possible, we try to be complete. When it is not feasible to provide a proof, a reference is given.
\end{document}