\documentclass[main.tex]{subfiles}

\begin{document}

\section{Introduction}

In 2008, Noah Snyder published \cite{snyderarticle} a result on a group parameter:

\hss

\begin{definition}
Let $G$ be a group of order $n$ and $V$ an irriducible representation over $\mathbb{C}$ of dimension $d$. Define $e$ to be the non-negative integer $n = d(d + e)$.
\end{definition}

\hss

If $e$ is small relative to $d$, then $G$ has a character of large degree.

His paper characterized groups possessing an $e \le 3$ and bounded the order of a group via this parameter. In particular, the following theorem is proven:

\hss

\begin{theorem}[Snyder \cite{snyderarticle}]
Let $G$ be a group of order $n$ with a simple $\C[G]$-module $V$ of dimension $d$ and $d(d+e) = n$. If $e > 1$, $n \le ((2e)!)^2$.
\end{theorem}

\hss

This begins a series of publications aiming to improve upon Snyder's bound.

The first artcle along this line of inquiry is by Isaacs \cite{isaacsarticle}. The relevant theorem is this:

\hss

\begin{theorem}[Isaacs \cite{isaacsarticle}] Let $G$ be a group of order $n$ and $d$ be the degree of some irreducible character of $G$. If $e > 1$, then $n \le Be^6$ for some universal constant $B$.
\end{theorem}

\hss

This result requires the simple group classification theorem for full generality. Interestingly, this result spawned another paper (citation needed) whose results were necessary to complete certain cases of Isaacs' proof.

Students of Isaacs, Christina Durfee and Sara Jensen, were the next to make headway on the problem. Their main result is the following:

\hss

\begin{theorem}[Durfee-Jensen \cite{durfeejensenarticle}]
For $e > 1$, we have the follwing bounds on $|G|$ in terms of $e$:
\begin{enumerate}
	\item If $e$ is divisible by two distinct primes, then $|G| < e^4 + e^3$.
	\item If $e$ is a prime power then $|G| < e^6 - e^4$.
	\item If $e$ is a prime, then $|G| < e^4 + e^3$.
\end{enumerate}
\end{theorem}

\hss

\noindent Since $e$ is an integer, this theorem has as an immediate corollary that $|G| < e^6 - e^4$ for all groups $G$. This removes the universal constant from Isaacs' bound and then some.

Most recently, Mark Lewis gave general conditions \cite{lewisarticle} for a group to satisfy the stronger bound $|G| \le e^4 - e^3$ and shows this to be the best possible.

\hss

\begin{theorem}[Lewis \cite{lewisarticle}] Let $G$ be a group with a nontrivial, abelian normal subgroup. Let $d$ be the degree of some irreducible character of $G$ and $|G| = d(d+e)$. If $e > 1$, then $d \le e^2 - e$ and $|G| \le e^4 - e^3$. This bound is best possible.
\end{theorem}

\hss

\noindent Notably, all solvable groups satisfy the conditions of this theorem. It thus applies to a large class of interesting groups.

In light of the aforementioned research, it is reasonable to consider the group theoretic analog. There is a close connection between characters and conjugacy classes. Sometimes, a theorem about one will be similar to a theorem about the other. A survey of such results can be found in \cite{larsenmalletieparticle}.

Our approach is to relate character degrees and conjugacy class sizes via the equation:
\begin{equation}\label{degreeidentity}
|G| = \sum_{i=1}^n d_i^2
\end{equation}
If we take $d_i$ to be the degree of an irreducible character where $i$ indexes $n$ character degrees, we obtain a well known identity \cite{gorensteinfinitegroups}. Conjugacy class sizes satisfy a similar identity:
\begin{equation*}
|G| = \sum_{i=1}^n |x_i^G|
\end{equation*}
where $|x_i^G|$ is a conjugacy class size and $i$ indexes $n$ distinct conjugacy class representitatives. Thus, if we take $d_i = \sqrt{|x_i^G|}$, then \ref{degreeidentity} is satisfied.

We then define an analogous parameter $e$ via $|G| = d(d+e)$ where $d$ is the square root of a conjugacy class size. This new parameter does not have the nice property of being an integer but it is non-negative. It also has the property that small values of $e$ correspond to tlarge conjugacy class sizes. The majority of this thesis will study this analogous parameter.

The remainder will focus on a more general version. The motivation of this comes from Durfee's dissertation (citation needed). She studies a generalization of Snyder's $e$ relative to a fixed normal subgroup. It is defined for a group $G$ with normal subgroup $N$. Let $\chi$ be an irreducible character of $G$ that restricts to a multiple of $\Theta$ on $N$. Define $d = \chi(1) / \Theta(1)$ and $|G/N| = d(d+e)$.

\noindent Durfee's relative parameter is a non-negative integer and, in the case of $N = 1$, becomes Snyder's parameter. Following this cue, we define an analog:

\hss

\begin{definition}
Let $N$ be a fixed normal subgroup of $G$ and let $x \in G$. Define $d = \sqrt{|x^G|/|x^N|}$ and $|G/N| = d(d+e)$.
\end{definition}

\noindent Here $d$ remains an integer but $e$ is real. The case of $N = 1$ yields our original analog. For this relative parameter, we will prove a bound and give a few classification theorem.

The rest of this paper will proceed with a few chapters of background material and then two chapters for our main results.

\end{document}