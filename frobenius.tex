\documentclass[main.tex]{subfiles}

\begin{document}

\section{Frobenius Groups}

Frobenius groups will play an important role in the results of this thesis. We present the required background here.

\hss

\begin{definition}
Let $G$ be a group and $N$ be a normal subgroup of $G$ with complement $A$. The group $G$ is a \emph{Frobenius group} if and only if $n^a \ne n$ for non-identity elements $n \in N$ and $a \in A$.
\end{definition}

\hss

An important example, for our purposes, will be the ``odd-order'' <cite> dihedral groups, that is, dihedral groups of order $2n$ for odd $n$. In this class of groups, we take $N$ to be the cyclic subgroup of order $n$ and $A$ to be any subgroup generated by an involution. Then $A$ acts on $N$ via conjugation such that $n^a = n^{-1}$ for non-identity elements $a \in A$ and $n \in N$. Moreover, since $N$ has odd order, $n = n^{-1}$ if and only if $n = 1$. Thus, the action of $A$ on $N$ satisfies the definition above and we may conclude that ``odd-order'' dihedral groups are Frobenius groups.

Similarly, generalized dihedral groups $G = B \rtimes C_2$ are Frobenius if $|B|$ is odd.

\hss

\subsection{Kernels and Complements}

\hss

From here on, the normal subgroup $N$ from the definition of Frobenius groups will be referred to as a \emph{Frobenius kernel} and its complement, $A$, will be a \emph{Frobenius complement}. These subgroups have many nice properties and relationships of which we will now mention a few.

\begin{lemma}\label{frobeniuscoprime}
Let $G$ be a Frobenius group with kernel $N$ and complement $A$. Then $|N| \equiv 1 \pmod{|A|}$ and, in particular, $|A|$ and $|N|$ are coprime.
\end{lemma}

\begin{proof}
Consider the $A$-orbits of $N$. By the Orbit-Stabilizer theorem, $|\mathcal{O}_n| = |A : C_A(n)|$ for all $n \in N$. But, $n^a \ne n$ for non-identity elements $n \in N$ and so $C_A(n) = 1$ unless $n = 1$. Exactly one orbit of $N$ is then of size $1$ and all others are of size $|A|$. We conclude that $|N| = 1 + k |A|$ for some positive integer $k$ and thus $|N| \equiv 1 \pmod{|A|}$.
\end{proof}

\begin{lemma}\label{complementconjugatesdisjoint}
If $G$ is a Frobenius group with kernel $N$ having complement $A$, then $A \cap A^g = 1$ for $g \in G - A$.
\end{lemma}

\begin{proof}
Suppose $A \cap A^x > 1$ for some $x \in G$. Since $G = AN$, we may write $x = an$ for some $a \in A$ and $n \in N$. Then $A \cap A^{an} = A \cap A^n > 1$ and there exists some non-identity $b^n \in A \cap A^n$. This gives us that $[b,n] = b^{-1}b^n \in A$. Since $N$ is normal, $[b, n] \in N$. But then $[b,n] \in A \cap N = 1$ and so $b$ must centralize $n$. Since $G$ is Frobenius we are forced to conclude $n = 1$ and that $x \in A$. Thus, $A \cap A^g = 1$ for $g \in G - A$.
\end{proof}

Other properties can be found in <cite> for the interested reader. We now turn our attention towards different characterizations of Frobenius groups.

\hss

\subsection{Transitive Permutation Groups}

\hss

Frobenius groups are \emph{transitive permutation groups} with the special property that only the identity fixes more than one letter but the subgroup fixing a letter is non-trivial. To see this, consider the right multiplication action of a Frobenius group $G$ on the cosets of its Frobenius complement $A$. We note that $G$ permutes the the set $\{Ag : g \in G \}$ and that the action is transitive. Hence, $G$ is a transitive permutation group via <cite>.

The more interesting properties are that each subgroup fixing a coset is non-trivial yet only the identity fixes more than one coset. We know that the subgroup $A^g$ of $G$ fixes the coset $Ag$ of $A$ for $g \in G$. This gives the former property. For the latter, the bulk of the work has already been done in Lemma \ref{complementconjugatesdisjoint}. If $x \in G$ fixes $A$ and $Ag$, then $x \in A \cap A^g = 1$.

Now, consider the elements not fixing any cosets of $A$. This set is:
$$X = G - \bigcup_{g \in G} A^g\text{.}$$
It turns out that $X \cup {1}$ is exactly the Frobenius kernel of $G$ and is, hence, normal. This is the content of the next theorem:

\begin{theorem}[Frobenius]
Let $G$ be a Frobenius group which permutes the set $S = \{1,...,n\}$. Let $A$ be the subgroup fixing $1 \in S$. Then the set:
$$N = (G - \bigcup_{g \in G} A^g) \cup \{1\}$$
is a normal subgroup of order $|G : A|$.
\end{theorem}

Unfortunately, no character free proof of this result has been discovered. The proof has been omitted here as the background material would significantly lengthen this thesis. For the motivated reader, one may consult Isaac's \emph{Algebra: A Graduate Course} <cite> for a chapter devoted to reaching this result as quickly as possible.

Moving on, we use this result to prove some other relevant characterizations.

\hss

\subsection{Characterizations}

\hss

\begin{lemma}\label{frobeniuscentralizers}
Let $G$ be a finite group and $N$ be a normal subgroup complemented by $A$. The following are equivalent:
\begin{enumerate}
	\item $G$ is Frobenius.
	\item $C_G(a) \le A$ for all non-identity $a \in A$.
	\item $C_G(n) \le N$ for all non-identity $n \in N$.
\end{enumerate}
\end{lemma}


\begin{lemma}\label{frobeniusnormalizer}
Let $A$ be a non-trivial subgroup of $G$. Then $G$ is Frobenius with complement $A$ if and only if $A$ is disjoint from its conjugates and $H$ is its own normalizer.
\end{lemma}

\end{document}