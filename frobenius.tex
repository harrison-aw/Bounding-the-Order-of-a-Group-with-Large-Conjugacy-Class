\documentclass[main.tex]{subfiles}

\begin{document}

\section{Frobenius Groups}

Frobenius groups will play an important role in the results of this thesis. We present the required background here.

\hss

\begin{definition}
Let $G$ be a group and $N$ be a normal subgroup of $G$ with complement $A$. The group $G$ is a \emph{Frobenius group} if and only if $n^a \ne n$ for non-identity elements $n \in N$ and $a \in A$.
\end{definition}

\hss

An important example, for our purposes, will be the ``odd-order'' dihedral groups, that is, dihedral groups of order $2n$ for odd $n$. In this class of groups, we take $N$ to be the cyclic subgroup of order $n$ and $A$ to be any subgroup generated by an involution. Then $A$ acts on $N$ via conjugation such that $n^a = n^{-1}$ for non-identity elements $a \in A$ and $n \in N$. Moreover, since $N$ has odd order, $n = n^{-1}$ if and only if $n = 1$. Thus, the action of $A$ on $N$ satisfies the definition above and we may conclude that ``odd-order'' dihedral groups are Frobenius groups.

Similarly, generalized dihedral groups $G = B \rtimes C_2$ are Frobenius if $|B|$ is odd.

\hss

\subsection{Kernels and Complements}

\hss

From here on, the normal subgroup $N$ from the definition of Frobenius groups will be referred to as a \emph{Frobenius kernel} and its complement, $A$, will be a \emph{Frobenius complement}. These subgroups have many nice properties and relationships of which we will now mention a few.

\begin{lemma}\label{frobeniuscoprime}
Let $G$ be a Frobenius group with kernel $N$ and complement $A$. Then $|N| \equiv 1 \pmod{|A|}$ and, in particular, $|A|$ and $|N|$ are coprime.
\end{lemma}

\begin{proof}
Consider the $A$-orbits of $N$. By the Orbit-Stabilizer theorem, $|\mathcal{O}_n| = |A : C_A(n)|$ for all $n \in N$. But, $n^a \ne n$ for non-identity elements $n \in N$ and so $C_A(n) = 1$ unless $n = 1$. Exactly one orbit of $N$ is then of size $1$ and all others are of size $|A|$. We conclude that $|N| = 1 + k |A|$ for some positive integer $k$ and thus $|N| \equiv 1 \pmod{|A|}$.
\end{proof}s

\begin{lemma}\label{frobeniusdisjoint}
If $G$ is a Frobenius group with kernel $N$ having complement $A$, then $A \cap A^g = 1$ for $g \in G - A$.
\end{lemma}

\begin{proof}
Suppose $A \cap A^x > 1$ for some $x \in G$. Since $G = AN$, we may write $x = an$ for some $a \in A$ and $n \in N$. Then $A \cap A^{an} = A \cap A^n > 1$ and there exists some non-identity $b^n \in A \cap A^n$. This gives us that $[b,n] = b^{-1}b^n \in A$. Since $N$ is normal, $[b, n] \in N$. But then $[b,n] \in A \cap N = 1$ and so $b$ must centralize $n$. Since $G$ is Frobenius we are forced to conclude $n = 1$ and that $x \in A$. Thus, $A \cap A^g = 1$ for $g \in G - A$.
\end{proof}

Other properties can be found in \cite{isaacsfinitegrouptheory}, \cite{gorensteinfinitegroups}, \cite{isaacsalgebra} for the interested reader.

\hss

\subsection{Frobenius' Theorem}

\hss

Frobenius groups can be viewed as those with a transitive action on a set satisfying the properties that:
\begin{enumerate}
	\item its stabilizers are non-trivial.
	\item only the identity fixes more than one letter.
\end{enumerate}
To see this, let $G$ be a Frobenius group with complement $A$ and consider its right multiplication action on the cosets of $A$. We know that this action is transitive, so all that remains to be shown is that it satisfies the aforementioned properties (see Theorem \ref{transitivegroup}).

The subgroup $A^g$ of $G$ fixes the coset $Ag$ of $A$, giving the former property. For the latter, the bulk of the work has already been done in Lemma \ref{frobeniusdisjoint}. If $x \in G$ fixes $A$ and $Ag$, then $x \in A \cap A^g = 1$.

Now, consider the elements not fixing any cosets of $A$:
$$X = G - \bigcup_{g \in G} A^g\text{.}$$
It turns out that $X \cup {1}$ is exactly the Frobenius kernel of $G$ and is, hence, normal. This is the content of the next theorem:

\begin{theorem}[Frobenius]
Let $G$ be a Frobenius group acting on a set $\Omega$. Let $A$ be the subgroup fixing $\alpha_0 \in \Omega$. Then the set:
$$N = (G - \bigcup_{g \in G} A^g) \cup \{1\}$$
is a normal subgroup of order $|G : A|$.
\end{theorem}

Unfortunately, no character free proof of this result has been discovered. The proof has been omitted here as the background material would significantly lengthen this thesis. For the motivated reader, one may consult Isaac's \emph{Algebra: A Graduate Course} <cite> for a chapter devoted to reaching this result as quickly as possible.

An useful corollary is this:

\begin{corollary}\label{frobeniuscorollary}
Let $G$ be a Frobenius group with kernel $N$ and complement $A$. If $g \in G - N$, then $g$ is contained in some conjugate of $A$.
\end{corollary}

Moving on, we use these results to prove some other relevant characterizations.

\hss

\subsection{Characterizations}

\hss

\begin{lemma}\label{frobeniuscentralizers}
Let $G$ be a finite group and $N$ be a normal subgroup complemented by $A$. The following are equivalent:
\begin{enumerate}
	\item $G$ is Frobenius.
	\item $C_G(a) \le A$ for all non-identity $a \in A$.
	\item $C_G(n) \le N$ for all non-identity $n \in N$.
\end{enumerate}
\end{lemma}

\begin{proof}
We will show that (1) is equivalent to (2) and then that (3) is equivalent to (1).

Suppose that $G$ is Frobenius. Let $a \in A$ be a non-identity element and take $x \in C_G(a)$. Then $a \in A \cap A^x$. Since the intersection is non-trivial, $x \in A$ and (2) holds.

Now, assume (2) and we show (1). Let $a \in A$ be a non-identity element. Then:
$$C_N(a) = N \cap C_G(a) \subseteq N \cap A = 1$$
where the subset relation holds by (2) and the final equality holds because $A$ is a complement of $N$.

Again, suppose $G$ is Frobenius. Let $n \in N$ be a non-identity element. Assume that $C_G(n) \not\subseteq N$. By Corollary \ref{frobeniuscorollary}, $a^g \in C_G(n)$ for some non-identity $a \in A$ and $g \in G$. If $m = n^{g^{-1}}$, then $a \in C_G(m)$. Since $N$ is normal, $m \in N$ and, because $G$ is Frobenius, we must have $m=1$. But then $n = 1$ --- a contradiction. Therefore, $C_G(n) \le N$.

Finally, we prove (1) from (3). Let $n \in N$ be a non-identity element. Then
$$C_A(n) = A \cap C_G(n) \subseteq A \cap N = 1$$
wehre the subeset relation holds by (3) and the latter equality because $A$ complements $N$. Thus, $G$ is Frobenius.
\end{proof}

\begin{lemma}\label{frobeniusnormalizer}
Let $A$ be a non-trivial subgroup of $G$. Then $G$ is Frobenius with complement $A$ if and only if $A$ is intersects its conjugates trivially and is its own normalizer.
\end{lemma}

\begin{proof}
Let $G$ be Frobenius with complement $A$. By Lemma \ref{frobeniusdisjoint}, we know that $A \cap A^g = 1$ for all $g \in G - A$.

Suppose $A \cap A^g = 1$ for all $g \in G - A$. Then $C_G(a) \le A$ for all non-identity $a \in A$.  According to Lemma \ref{frobeniuscentralizers}, this is equivalent to the statement that $G$ is Frobenius.

All that remains to be shown is that $A$ is self-normalizing. Let $x \in N_G(A)$. Then $A \cap A^x = A$. Lemma \ref{frobeniusdisjoint} gives that $x \in A$. It follows that $N_G(A) = A$ and this completes the proof.
\end{proof}

\end{document}