\documentclass[main.tex]{subfiles}

\begin{document}

\chapter{Dihedral Groups}

Dihedral groups will play an important role in later results. While some familiarity with their structure is assumed, we will review their relevant properties.

\hss

We begin with our definition:

\hss

\begin{definition}
A \emph{dihedral group} is a group $D$ having a cyclic subgroup $C$ of index 2 such that $D - C$ contains only elements of order 2.
\end{definition}

\begin{remark}
Both the cyclic subgroup of order 2 and the Klein 4-group are dihedral. For the former, take $C = 1$. In the latter, all non-identity elements are involutions and any subgroup of order 2 will suffice for $C$.
\end{remark}

\hss

Since $|D : C| = 2$, we know that $C \normal D$ and $D$ has even order. Dihedral groups also have two generators: There exists a $c \in C$ such that $\generate{c} = C$. We also have that $D - C$ is a coset of $C$. If $t \in D - C$, then $Ct = D - C$ and $\generate{c,t} = D$. Another important property of dihedral groups is that, for $c \in C$ and $t \in D - C$, conjugation of $c$ by $t$ yields the inverse of $c$, that is $c^t = c^{-1}$. Showing this is slightly more involved but still follows quickly from our definition. We know that $ct \in D - C$, so $ct$ must be an involution. Then $(ct)^2 = ctct = 1$. Left multiplication by $c^{-1}$ now gives the desired relation: $tct = c^t = c^{-1}$. This last property is often used as a defining characteristic in the presentation of dihedral groups: 
$$D = \generate{c, t : c^n = 1, t^2 = 1, t^{-1}ct = c^{-1}}$$
where $|D| = 2n$ for some $n \in \Z$. That presentation is the same as what we have defined above since $\generate{c}$ is a cyclic group and $t$ cannot be in $\generate{c}$. Furthermore, $\generate{c} t$ contains only involutions because $(c^k t) \cdot (c^k t) = c^n \cdot (t^{-1} c^k t) = c^n c^{-n} = 1$.

Another way view to dihedral groups is as the semi-direct product of $C_n \rtimes_\varphi C_2$ where $C_n$ denotes the cyclic group of order $n$ and $\varphi : C_2 \to Aut(C_n)$ is the map that sends $1_{C_2}$ to the identity automorphism and $C_2$'s other element to the automorphism $c \mapsto c^{-1}$. From this view, we have some element $(c, 1) \in C_n \rtimes C_2$ that generates the subgroup identified with $C_n$ and another element $(1, t) \in C_n \rtimes C_2$ that generates the subgroup identified with $C_2$. Furthermore, $(1, t)$ has the desired conjugation action on $(c, 1)$, i.e. $(c,1)^{(1,t)} = (c^{-1}, 1)$. Finally, we note that $(c, 1)$ and $(1, t)$ comprise a complete set of generators for $C_n \rtimes C_2$. Then 
\begin{align*}
C_n \rtimes_\varphi C_2 = \generate{(c, 1), (1, t) : &(c, 1) = (1, 1), \\
&(1, t)^2 = (1, 1), \\
&(1, t^{-1})(c, 1)(1, t) = (c^{-1}, 1)}
\end{align*}
for which there is an obvious isomorphism to $\generate{c, t : c^n = 1, t^2 = 1, t^{-1} c t = c^{-1}}$.

\hss

While not strictly necessary for our results, we will frequently make use of the fact that, up to isomorphism, there is exactly one dihedral group of a given order.

\begin{proposition}
All dihedral groups of order $2n$ are isomorphic.
\end{proposition}

\begin{proof}
Let $D_0$ and $D$ be dihedral groups of order $2n$. We seek an isomorphism $\varphi: D_0 \to D$. Let $D_0 = \langle c_0, t_0 \rangle$ and $D = \langle c, t \rangle$ such that $|c_0| = |c| = n$ and $|t| = |t_0| = 2$. Define $\varphi(c_0^k t_0^m) = c^k t^m$ for $0 \le k < n$ and $0 \le m < 2$. Then $\varphi(c_0^k) = c^k$ and $\varphi(t_0) = t$. We now show $\varphi$ to be a homomorphism. There are 4 cases:
\begin{enumerate}
	\item $\varphi(c_0^{k_1} c_0^{k_2}) = c^{k_1 + k_2} = c^{k_1} c^{k_2} = \varphi(c_0^{k_1}) \varphi(c_0^{k_2})$.
	\item $\varphi(c_0^{k_1} (c_0^{k_2} t_0)) =  c^{k_1+k_2} t = c^{k_1} (c^{k_2} t) = \varphi(c_0^{k_1}) \varphi(c_0^{k_2} t_0)$.
	\item $\varphi((c_0^{k_1} t_0) c_0^{k_2}) = c^{k_1 - k_2} t = (c^{k_1} t) c^{k_2} = \varphi(c_0^{k_1} t_0) \varphi(c_0^{k_2})$.
	\item $\varphi((c_0^{k_1} t_0) (c_0^{k_2} t_0)) = c^{k_1 - k_2} = (c^{k_1} t) (c^{k_2} t) = \varphi(c_0^{k_1} t_0) \varphi(c_0^{k_2} t_0)$.
\end{enumerate}
The above gives that $\varphi$ respects the multiplication of $D_0$ and is, hence, a homomorphism. Because every $d \in D$ can be written $d = c^k t^m$ with $0 \le k < n$, $0 \le m < 2$ and that $\varphi(c_0^k t_0^m) = c^k t^m$,  $\varphi$ is surjective. To show injectivity, suppose $\varphi(c_0^k t_0^m) = \varphi(c_0^l t_0^r)$ for $0 \le k, l < n$ and $0 \le m, r < 2$. Then $c^k t^m = c^l t^r$ which implies that $l \equiv k \pmod n$ and $r \equiv m \pmod 2$. By the bounds imposed on $k$, $m$, $l$, and $r$, we must have $k = l$ and $m = r$.  Then $c_0^k t_0^m = c_0^l t_0^r$. We conclude that $\varphi$ is an isormorphism and the proof is complete.
\end{proof}

\hss

\section{Conjugacy Classes}

\hss

We now discuss the conjugacy class structure of dihedral groups. The number and sizes of conjugacy classes are completely determined by the order of a dihedral group. In the following propositions, let $D$ be a dihedral group and $C \normal D$ be cyclic of index 2.

\begin{proposition}\label{cclassstructure}
If $c \in C$, then $c^D = \{c, c^{-1}\}$.
\end{proposition}

\begin{proof}
Since $C$ is abelian, $c^C = \{c\}$. Then we must only consider the action of the involutions in $D - C$. Let $z$ be a generator of $C$ and $t$ be an involution in $D - C$. Then every element of $D - C$ is of the form $z^k t$ for some non-negative integer $k$. For any $c \in C$:
$$t z^{-k} c z^k t = t c t = c^{-1} \text{.}$$
Thus every involution of $D - C$ sends an element of $C$ to its inverse via conjugation. It follows that $c^D = \{c, c^{-1}\}$ for all $c \in C$.
\end{proof}

The remaining conjugacy classes have a different structure depending on the parity of the $|C|$.  Note also that if $|C|$ is even, it contains an involution which must have a conjugacy class consisting only of itself. The following lemma is easy, but necessary for the next proposition. It is presented here for completeness.

\begin{lemma}\label{paritylemma}
Let $n$ be an even integer and $k$ be any integer. Then $k$ and $k \bmod n$ have the same parity.
\end{lemma}

\begin{proof}
Suppose $k \ge 0$ and let $k \bmod n = r$. Then $r = k - n \cdot m$ for some integer $m$. Immediately, $r$ and $k$ have the same parity since $n \cdot m$ must be even. If $k < 0$, then $r = -k - n \cdot m$ and the same logic gives that $k$ and $k \bmod n$ have the same parity.
\end{proof}

\begin{proposition}\label{evendihedralclass}
Let $D$ have order $2n$, $n$ even. Then $D$ has 3 distinct conjugacy classes of involutions: two of size $n/2$ and one of size $1$.
\end{proposition}

\begin{proof}
Since $C$ has even order $n$, it contains a unique involution which lies in its own conjugacy class by the Proposition \ref{cclassstructure}. The remaining involutions are of the form $z^k t$ ($k \in \Z$) where $\generate{z} = C$ and $t \in D - C$. Consider the following:
\begin{align*}
z^{-m} (z^k t) z^m &= z^{-m} (z^k t) z^m t t = z^{k-2m} t \\
t z^{-m} (z^k t) z^m t &= t z^{k-m} t z^m t= z^{2m-k} t \text{,}
\end{align*}
with $m$ an arbitrary integer. Note that $k$, $k-2m$, $2m-k$ are all of the same parity. By Lemma \ref{paritylemma}, we must have that $k \bmod n$, $k-2m \bmod n$, and $2m-k \bmod n$ share the same parity as well. Then the conjugacy classes of involutions in $D - C$ are $\{z^k t : k \text{ is even}\}$ and $\{z^k t : k \text{ is odd}\}$. Each class must contain exactly half of the involutions in $D - C$ so they are both of size $n/2$.
\end{proof}

Now for the odd case, we have:

\begin{proposition}\label{odddihedralclass}
Let $D$ have order $2n$, $n$ odd. Then $D$ has exactly one class of involutions of size $n$.
\end{proposition}

\begin{proof}
Each involution of $D$ generates a Sylow 2-subgroup of order 2. All of these are distinct and must be conjugate. Then all involutions of $D$ are in the same conjugacy class.
\end{proof}

\hss

\section{Factor Groups}

\hss

We will show that dihedral groups have the property that every factor group is dihedral. Before we get to this, let us discuss the normal subgroups of dihedral groups.

\begin{proposition}
If $N \normal D$ is not a subgroup of $C$, then $N = D$ or $|D : N| = 2$.
\end{proposition}

\begin{proof}
Let $D$ have order $2n$, $\generate{z} = C$, $t \in D - C$, and $N \normal D$. Suppose $t \in N$. By normality, $t^D \subseteq N$. If $n$ is odd, then Proposition \ref{odddihedralclass} gives us that $z t \in N$. So $z t \cdot t = z \in N$ and thus $N = D$. If $n$ is even, then $z^2 t \in N$ by Proposition \ref{evendihedralclass}. Then $\generate{z^2} < N$ and $\generate{z^2} t \subset N$. We infer that $|N| \ge n$. But then, $|D : N| \le 2$ and the proposition holds.
\end{proof}

\begin{proposition}
If $D$ is dihedral and $N \normal D$, then $D/N$ is dihedral.
\end{proposition}

\begin{proof}
There are two cases, either $N$ contains an involution from $D - C$  or $N \le C$. In the former case, we know by the previous proposition that either $N = D$ or $|D/N| = 2$. Since both the trivial group and the cyclic group of order 2 are dihedral, we must only show the remaining case.

Let $N \le C$. Let $\bar{ \cdot }$ denote the canonical mapping of $D \mapsto D/N$ where $\overline{D} = D/N$. We know that $\overline{C}$ is cyclic and that $|\overline{D} : \overline{C}| = |D : C| = 2$. Furthermore, for all $t \in D - C$, $\overline{t} \in \overline{D} - \overline{C}$ and $\overline{t}^2 = \overline{tt} = 1$. Then $\overline{D}$ is dihedral.

\end{proof}

\hss

\section{Generalized Dihedral Groups}

\hss

It is possible to extend the notion of a dihedral group by relaxing the cyclic condition on the subgroup of index 2.

\hss

\begin{definition}
A \emph{generalized dihedral group} is a group $G$ with an abelian subgroup $B$ of index 2 such that $G - B$ contains only involutions.
\end{definition}

\hss

As in the case of dihedral groups, we may view $G$ under a semi-direct product construction where $G = B \rtimes_\varphi C_2$ where $B$ is an abelian group, $C_2$ is the cyclic group of order 2, and $\varphi$ sends the non-identity element of $C_2$ to the inverse map ($b \mapsto b^{-1}$). 

It is not suprising then that this class of groups shares a similar conjugacy class structure with the usual dihedral groups in certain cases.

\begin{proposition}\label{dihedralinvolutionclass}
If $G$ is generalized dihedral of order $2n$ with odd $n$, then $G$ has a single conjugacy class of involutions.
\end{proposition}

\begin{proof}
This follows immediately from the Sylow conjugacy theorem.
\end{proof}

We cannot, however, say much about the conjugacy classes of involutions if $n$ is even. Consider the direct product of 3 cyclic groups of order 2: $C_2 \times C_2 \times C_2$. This group satisfies the conditions of generalized dihedral groups but is abelian and every element is an involution! Thus, every conjugacy class consists of only a single element. 

\end{document}