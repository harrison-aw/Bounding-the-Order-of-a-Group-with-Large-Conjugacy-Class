\documentclass[main.tex]{subfiles}

\begin{document}

\section{Dihedral Groups}

Dihedral groups are used throughout the results of this document. While some familiarity with their structure is assumed, the properties and theorems necessary for our results will be presented here.

\begin{definition}
A \emph{dihedral group} is a group $D$ having a cyclic subgroup $C$ of index 2 such that $D - C$ contains only elements of order 2.
\end{definition}

\begin{remark}
Both the cyclic subgroup of order 2 and the Klein 4-group are dihedral. For the former, take $C = 1$. In the latter, all non-identity elements are involutions and any subgroup of order 2 will suffice for $C$.
\end{remark}

An easy consequence of this definition is that $C \normal D$ and that $D$ has even order. It is also not hard to see that $D$ is generated by two elements. There exists a $c \in C$ such that $\generate{c} = C$ and that $D - C$ is a coset of $C$. If $t \in D - C$, then $Ct = D - C$ and $\generate{c,t} = D$.  Another basic property of dihedral groups is that, for $c \in C$ and $t \in D - C$, conjugation of $c$ by $t$ yields the inverse of $c$, that is $c^t = c^{-1}$. Showing this is slightly more involved but still follows readily from our definition. We know that $ct \in D - C$, so $ct$ must be an involution. Then $ct^2 = ctct = 1$. Left multiplication by $c^{-1}$ now gives the desired relation: $tct = c^t = c^{-1}$.


\subsection{Conjugacy Classes}

The conjugacy classes of dihedral groups will play an important role in later results. In the following propositions, let $D$ be a dihedral group and $C \normal D$ be cyclic of index 2.

\begin{proposition}
If $c \in C$, then $c^D = \{c, c^{-1}\}$.
\end{proposition}

\begin{proof}
Since $C$ is abelian, $c^C = \{c\}$. Then we must only consider the action of the involutions in $D - C$. Let $z$ be a generator of $C$ and $t$ be an involution of in $D - C$. Then every element of $D - C$ is of the form $z^k t$ for some non-negative integer $k$. Then for any $c \in C$:
$$t z^{-k} c z^k t = t c t = c^{-1} \text{.}$$
This shows that every involution of $D - C$ sends an element of $C$ to its inverse via conjugation. It follows that $c^D = \{c, c^{-1}\}$ for all $c \in C$.
\end{proof}

This result gives that all involutions of $C$ have a conjugacy class of size 1. The next result clarifies this and explains the rest of the conjugacy class structure.

\begin{proposition}
Let $D$ have order $2n$. If $n$ is even, then $D$ has 3 distinct conjugacy classes of involutions: two of size $n/2$ and one of size $1$. If $n$ is odd, then $D$ has exactly one class of involutions.
\end{proposition}

\begin{proof}
We begin with the even case. Since $C$ must have even order, it contains an involution. As we remarked above, this element must belong to its own single-element conjugacy class. The remaining involutions are of the form $z^k t$ where $\generate{z} = C$ and $t \in D - C$. Consider the following:
\begin{align*}
z^{-m} (z^k t) z^m &= z^{-m} (z^k t) z^m t t = z^{k-2m} t \\
t z^{-m} (z^k t) z^m t &= t z^{k-m} t z^m t= z^{2m-k} t
\end{align*}
where $m$ is an arbitrary integer. We must have that $k \bmod n$, $k-2m \bmod n$, and $2m-k \bmod n$ share the same parity for all integers $k,m \in \mathbb{Z}$.
To see this, let $n = 2n_0$ and $k = 2k_0$ for integers $n_0, k_0$. Then 
\begin{align*}
k \bmod n &= k - nq \\
&= 2k_0 - 2n_0 q  \\
&= 2(k_0 - n_0 q)
\end{align*}
 for some integer $q$. Simliar arguments can be made for the other operations:
\begin{align*}
2m - k \bmod n &= (2m-k) - n q_1 \\
&= 2[(m-k_0) - n_0 q_1 ]
\end{align*}
\begin{align*}
k - 2m \bmod n &= (k-2m) - n q_2\\
&= 2[(k_0-m) - n_0 q_2]
\end{align*}
where $q_1$ and $q_2$ are both integers.

If $k = 2k_0 + 1$ then, for integers $q, q_1, q_2$, we have:
\begin{align*}
k \bmod n &= k - n q \\
&= 2[k_0 - n_0 q] + 1
\end{align*}
\begin{align*}
2m - k \bmod n &= 2m - k - n q_1 \\
&= 2[m - k_0 - n_0 q_1] + 1
\end{align*}
\begin{align*}
k - 2m \bmod n &= k - 2m - n q_2 \\
&= 2[k_0 - m - n_0 q_2] + 1
\end{align*}

Then the conjugacy classes of involutions in $D - C$ are split into two distinct sets:
\begin{align*}
&\{z^k t : k \text{ is even}\} \text{ and} \\
&\{z^k t : k \text{ is odd}\}
\end{align*}

This completes the $n$ even case since exactly half of the elements of $D - C$ must belong to each of these sets, i.e. $n/2$ of them.

Now we suppose that $D$ has order $2n$ for odd $n$. In each case, an involution generates a Sylow $2$-subgroup of $D$. We know by the Sylow theorems that all such groups must be conjugate. Then all involutions of $D$ fall into the same conjugacy class.
\end{proof}

\subsection{Factor Groups}

Dihedral groups have the peculiar property that every factor group is dihedral. First, we consider normal subgroups of dihedral groups.

\begin{proposition}
If $N \normal D$ is not a subgroup of $C$, then $N = D$ or $|D : N| = 2$.
\end{proposition}

\begin{proof}
Let $N \normal D$ and $t \in N$ where $t \in D - C$. Then $t^D \subset N$. It follows that $z^2 \in N$ where $\generate{z} = C$. Let $D$ have order $2n$. If $n$ is odd, then $\generate{z^2} = C$ and $N = D$. If $n$ is even, then $\generate{z^2}$ is of order $n/2$ and $|N| \ge n$. We conclude that $|D : N| \le 2$ and the proposition holds.
\end{proof}

\begin{proposition}
If $D$ is dihedral and $N \normal D$, then $D/N$ is dihedral.
\end{proposition}

\begin{proof}
There are two cases, either $N$ contains an involution from $D - C$  or $N \le C$. In the former case, we know by the previous proposition that either $N = D$ or $|D/N| = 2$. Since both the trivial group and the cyclic group of order 2 are dihedral, we must only show the remaining case.

Let $N \le C$. Let $\overline{D} = D/N$. We know that $\overline{C}$ is cyclic and that $|\overline{D} : \overline{C}| = |D : C| = 2$. Furthermore, for all $t \in D - C$, $\overline{t} \in \overline{D} - \overline{C}$ and $\overline{t}^2 = \overline{tt} = 1$. Then $\overline{D}$ is dihedral.

\end{proof}

\subsection{Generalized Dihedral Groups}

It is possible to extend the notion of a dihedral group by relaxing the cyclic condition on the subgroup of index 2.

\begin{definition}
A \emph{generalized dihedral group} is a group $G$ with an abelian subgroup $B$ of index 2 such that $G - B$ contains only involutions.
\end{definition}

This class of groups shares a few properties with the usual dihedral groups. In particular, there are some similarities in conjugacy class structure.

\begin{proposition}
If $G$ is generalized dihedral of order $2n$ with odd $n$, then $G$ has a single conjugacy class of involutions.
\end{proposition}

\begin{proof}
This follows immediately from the Sylow conjugacy theorem.
\end{proof}

We cannot say much about the conjugacy classes of involutions if $n$ is even. Consider the direct product of 3 cyclic groups of order 2: $C_2 \times C_2 \times C_2$. This group satisfies the conditions of generalized dihedral groups but is abelian and every element is an involution! Then every conjugacy class consists of only a single element.

\end{document}