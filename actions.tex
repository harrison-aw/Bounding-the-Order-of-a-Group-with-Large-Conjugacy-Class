\documentclass[main.tex]{subfiles}

\begin{document}

\section{Group Actions}

We will need a few properties of group actions in later results. Here we present the required background.

\begin{definition}
Let $G$ be a group and $\Omega$ be a set. A \emph{group action} of $G$ on $\Omega$ is a mapping from $\Omega \times G$ to $\Omega$ that satisfies:
\begin{align*}
\alpha \cdot 1 &= \alpha \\
(\alpha \cdot g) \cdot h &= \alpha \cdot (gh) \\
\end{align*}
for $\alpha \in \Omega$ and $g, h \in G$.
\end{definition}

For a simple example, let $\Omega = G$ and define $\alpha \cdot g = \alpha g$. Then $\alpha \cdot 1 = \alpha 1 = \alpha$ and $(\alpha \cdot g) \cdot h = \alpha g h = \alpha \cdot (gh)$ and so $G$ acts on itself by right multiplication. Conjugation gives a more interesting example. Again take $\Omega = G$ but let $\alpha \cdot g = \alpha^g$. Now, $\alpha \cdot 1 = \alpha^1 = \alpha$ and $(\alpha \cdot g) \cdot h = (\alpha^g)^h = \alpha^{gh} = \alpha \cdot (gh)$.

To extend this example, let us introduce the concept of an orbit.

\hss

\begin{definition}
The \emph{orbit} of $\alpha \in \Omega$ is the set:
$$\mathcal{O}_\alpha = \{ \beta \in \Omega : \alpha \cdot g = \beta \text{ for some } g \in G \}\text{.}$$
\end{definition}

\hss

Orbits partition $\Omega$ into distinct classes. This is show by analyzing the implied relation: $\alpha \sim \beta$ iff $\exists g \in G : \alpha \cdot g = \beta$. A few quick calculations show it to be an equivalence relation:
\begin{align*}
&\alpha \cdot 1 = \alpha &\text{(reflexive)} \\
&\text{if } \alpha \cdot g = \beta \text{, then } \beta \cdot g^{-1} = \alpha &\text{(symmetric)} \\
&\text{if } \alpha \cdot g = \beta \text{ and } \beta \cdot h = \gamma \text{, then } \alpha \cdot gh = \gamma  &\text{(transitive)}
\end{align*}
where $\alpha, \beta, \gamma \in \Omega$ and $g, h \in G$.

For the conjugation action discussed above, orbits are conjugacy classes. That is:
$$\mathcal{O}_g = g^G = \{h \in G : g^{g_1} = h \text{ for some } g_1 \in G \} \text{.}$$
 Using properties of orbits, we can easily determine the size of conjugacy classes. This calculation involves the notion of a \emph{stabilizer}.

\hss

\begin{definition}
The \emph{stabilizer} of $\alpha \in \Omega$ is the group 
$$G_\alpha = \{g \in G : \alpha \cdot g = \alpha \} \text{.}$$
\end{definition}

\hss

It should be clear that $G_\alpha \le G$: if $\alpha \cdot g = \alpha$ and $\alpha \cdot h = \alpha$ then $\alpha \cdot gh = (\alpha \cdot g) \cdot h = \alpha$. For the conjugation action mentioned above, the stabilizer of $g \in G$ is exactly $C_G(g)$, the centralizer of $g$. To see this, let $h \in G_g$. Then $g \cdot h = h^{-1} g h = g$ iff $gh = hg$ and so $h \in C_G(g)$. The same argument suffices to prove $h \in C_G(g)$ implies $h \in G_g$. Then $G_g = C_G(g)$ for the conjugation of $G$ on itself.

We now determine the size of orbits and conjugacy classes.

\begin{theorem}[The Fundamental Counting Principle, p.5]
Let $H = G_\alpha$ and $\Lambda = \{H x : x \in G\}$. There exists a bijection $\theta : \Lambda \to \mathcal{O}_\alpha$ such that $\theta(H x) = \alpha \cdot x$. In particular,
$|\mathcal{O}_\alpha| = |G : H|$.
\end{theorem}

\begin{proof}
We first show that if $H x = H y$, then $\alpha \cdot x = \alpha \cdot y$. Since $y \in Hx$, $y = h x$ for some $h \in H$. Then:
$$\alpha \cdot y = \alpha \cdot h x = (\alpha \cdot h) \cdot x = \alpha \cdot x$$
where the final equality holds because $H$ is the stabilizer of $\alpha$. This shows that each coset of $H$ sends $\alpha$ to exactly one element of $\mathcal{O}_\alpha$.

Then we can define the map $\theta: \Lambda \to \mathcal{O}_\alpha$ by $\theta(H x) = \alpha \cdot x$. Since every element of $G$ lies in some coset of $H$, it follows from the definition of an orbit that $\theta$ is surjective. This leaves injectivity. Suppose $\theta(H x) = \theta(H y)$. Then:
$$ \alpha = (\alpha \cdot x ) \cdot x^{-1} = (\alpha \cdot y) \cdot x^{-1} = \alpha \cdot y x^{-1} \text{.}$$
Then $y x^{-1} \in H$ and so $y \in Hx$. It follows that $\theta$ is a bijection. Thus, the size of $\mathcal{O}_\alpha$ is equal to the number of cosets of $H$, i.e. $\mathcal{O}_\alpha = |G : H|$.

\end{proof}

Immediately, we have:

\begin{corollary}
The size of a conjugacy class $g^G$ is $|G : C_G(g)|$.
\end{corollary}


\subsection{Transitive Actions}

Transitive actions have the special property that they produce exactly one orbit.

\hss

\begin{definition}
Let $G$ be a group acting on the set $\Omega$. We call this a \emph{transitive action} iff it has exactly one orbit.
\end{definition}

\hss

Equivalently, transitive actions are those that, for all pairs $\alpha, \beta \in \Omega$, there exists a $g \in G$ such that $ \alpha \cdot g = \beta$. This view is perhaps more suggestive but, in our results, the former definition will prove more immediately useful.

An easy transitive action is to let a group $G$ act on itself by right multiplication. Take $g \in G$. Then for any $h_1 \in G$ there is an $h_2 \in g$ such that $g h_2 = h_1$. Then all elements of $G$ are in $\mathcal{O}_g$ and the right multiplication action is transitive.

Another example, to build off previous material, is to take $G$ to be a generalized dihedral group such that $|G| = 2n$ with $n$ odd. Then $G$ acts transitively on its involutions.

\begin{lemma}
Let $G$ be a finite group and $H$ be a cyclic subgroup of prime order $p$ such that $C_G(h) \le H$ for all non-identity $h \in H$. If $x \in N_G(H) - H$ and $|x| \ge p$, then $\generate{x}$ acts transitively via conjugation on the non-identity elements of $H$.
\end{lemma}

\begin{proof}
Let $\generate{h} = H$ and consider $h^x$. We must have $h^x = h^n$ for some integer $1 < n < p$. Then $h^{x^k} = h^{n^k}$ where $h^{n^k} = h^{n^k \bmod p}$. But then the orbit $\mathcal{O}_h = \{h^1, ... , h^{p-1} \}$ is precisely the non-identity elements of $H$. Thus, $\generate{x}$ acts transitively.
\end{proof}

\end{document}