\documentclass[main.tex]{subfiles}

\begin{document}

\chapter{Appendix A: The Largest Conjugacy Class of the Symmetric Group}

Conjugation in the symmetric group is well-understood. Here we build off the conjugation properties of $S_n$ and show that its largest conjugacy class is that of the $(n-1)$-cycles.

To begin, we review a few basic facts about $S_n$.

\begin{theorem}[\cite{symmetricconjugation}]
Disjoint cycles of $S_n$ commute.
\end{theorem}

\begin{theorem}[\cite{symmetricconjugation}]
Every permutation in $S_n$ has a cycle decomposition that is unique up to ordering of the cycles and up to a cyclic permutation of the elements within each cycle.
\end{theorem}

\begin{theorem}[\cite{symmetricconjugation}]\label{permutationclasssize}
Suppose $\sigma \in S_n$, and let $m_1, m_2, \ldots m_r$ be the distinct integers (including 1 if applicable) in the cycle type of $\sigma$, and let there be $k_i$ cycles of order $m_i$ in $\sigma$. (Thus $\sum k_i m_i = n$.) Then $\sigma$ has conjugacy class of size:
$$\frac{n!}{\prod_{i=1}^r (k_i!m_i^{k_i})}\text{.}$$
\end{theorem}

The previous theorem tells us precisely the size of each conjugacy class in $S_n$. However, its form does not make it amenable to standard analysis techniques. Instead, we take another approach.

\begin{lemma}\label{permutationcentralizer}
Let $\sigma \in S_n$. Then $|C_{S_n}(\sigma)| \ge n - 1$.
\end{lemma}

\begin{proof}
If $\sigma$ is the identity, it has centralizer equal to $S_n$. Assume then that $\sigma \in S_n$ is a non-identity element. Then the permutation $\sigma$ has a unique cycle decomposition into $m$ cycles $\sigma = \sigma_1 \sigma_2 \ldots \sigma_m$ (not including $1$-cycles) where each $\sigma_i$ is disjoint from $\sigma_j$ for all $j \ne i$. Let $k_i$ denote the cycle length of $\sigma_i$ and $\kappa = \sum_{i = 1}^m k_i$. Note that $\kappa \le n$. Consider the elements that commute with $\sigma$: Each subgroup $\generate{\sigma_i}$ commutes with $\sigma$ as does each cycle disjoint from $\sigma$. Together then, we have
$$|C_{S_n}(\sigma)| \ge \left(\sum_{i=1}^m k_i - 1\right) + (n - \kappa) = n - m$$
where the summation does not count identity elements. But then $\sigma$ also commutes with
\begin{align*}
&\sigma_1 \sigma_2 \ldots \sigma_m \text{,} \\
&\sigma_1 \sigma_2 \ldots \sigma_{m-1} \text{,} \\
&\vdots \\
&\sigma_1 \sigma_2
\end{align*}
which yields another $m-1$ elements not already counted. Thus $|C_{S_n}(\sigma)| \ge n-1$.
\end{proof}

\begin{corollary}\label{maxclasssize}
The symmetric group $S_n$ can have no conjugacy class larger than $n!/(n-1)$.
\end{corollary}

\begin{theorem}
The largest conjugacy class in $S_n$ is of size $n!/(n-1)$.
\end{theorem}

\begin{proof}
Let $\sigma \in S_n$ be an $(n-1)$-cycle. Its cycle type is $[(n-1), 1]$ and it has only one cycle of each order. Then $\sigma$ has class size:
$$\frac{n!}{(1!(n-1)^1)(1!1^1)} = \frac{n!}{n-1}$$
by Theorem \ref{permutationclasssize}. By Corollary \ref{maxclasssize}, this is the largest conjugacy class.
\end{proof}

\end{document}