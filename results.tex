\documentclass[main.tex]{subfiles}

\begin{document}

\section{Main Results: An Analog to Snyder's Parameter}

\hss

\begin{definition}
Let $G$ be a finite group. For $x \in G$ let $c(x) = |C_G(x)|$ and $k(x) = |G : C_G(x)|$. Then we define the parameter $e$ as follows:
$$e = min\{(c(x) - 1) \cdot \sqrt{k(x)} : x \in G\}\text{.}$$
\end{definition}

\hss

This parameter is small for groups with large conjugacy class. To see this, let us define $e$ in an alternate way. Let $G$ be a group. If the set $\{x_i \in G : 1 \le i \le n\}$ is a complete set of representatives of the conjugacy classes of $G$, then:
$$\sum_{i=1}^{n} |G : C_G(x_i)| = |G|\text{.}$$
Now, with a cue from Snyder's work, let $d_i = \sqrt{|G : C_G(x_i)}$ for some $1 \le i \le n$. This allows us to have a value that relates conjugacy classes to the group order in a manner similar to character degrees:
$$\sum_{i=1}^{n} d_i^2 = |G|\text{.}$$
Once we have this, we may define $e$ in the same manner it was originally introduced: $|G| = d_i(d_i + e)$. Manipulating this equality and requiring $e$ to be minimal verifies that the parameter mentioned at the start of the section is the very same one that we have just explained. With this, it is now clear that small values of $e$ relate to large conjugacy class sizes, that is large $d_i^2$.

\end{document}