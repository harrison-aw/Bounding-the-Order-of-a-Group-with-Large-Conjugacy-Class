\documentclass[main.tex]{subfiles}

\begin{document}

\section{Main Results: An Analog to Snyder's Parameter}

\hss

\begin{definition}
Let $G$ be a finite group. For $x \in G$ let $c(x) = |C_G(x)|$ and $k(x) = |G : C_G(x)|$. We define the parameter $e$ as follows:
$$e = min\{(c(x) - 1) \cdot \sqrt{k(x)} : x \in G\}\text{.}$$
\end{definition}

\hss

This parameter is small for groups with a large conjugacy class. Let $x \in G$ be such that $e = (c(x) - 1) \cdot \sqrt{k(x)}$. Consider the following:
\begin{align*}
e &= (c(x) - 1) \cdot \sqrt{k(x)} \\
 &= c(x)\frac{k(x)}{\sqrt{k(x)}} - \sqrt{k(x)} \\
 &= \frac{|G|}{\sqrt{k(x)}} - \sqrt{k(x)}\text{.}
\end{align*}
If we let $d = \sqrt{k(x)}$, we arrive at the equation $|G| = d (d + e)$. Since $k(x)$ is precisely the size of the conjugacy class containing $x$, for a fixed $G$, when the conjugacy class of $x$ is very large, $e$ is very small. This behavior is similar to Snyder's parameter with character degrees. Moreover, if we were to replace our $d$ by the degree of an irreducible character, we obtain Snyder's original parameter. With this, we now follow Snyder's example and classify groups where $e$ is 0,1, or 2.

\begin{theorem}
Let $G$ be a finite group.
\begin{enumerate}
	\item $e = 0$ if and only if $G$ is trivial.
	\item $e = 1$ if and only if $G$ is $C_2$.
	\item $e = 2$ if and only if $G$ is $C_3$.
\end{enumerate}
\end{theorem}

\begin{proof}
For the following, let $x \in G$ be an element such that $e = (c(x) - 1) \cdot \sqrt{k(x)}$.


Suppose $e = 0$. Then $c(x) = 1$. Only the trivial group has a trivial centralizer. This is sufficient for (1).

Let $e = 1$. Then $c(x) = 2$ and $k(x) = 1$. It must be the case that $|G| = 2$ and so $G$ is $C_2$. Reversing this argument proves (2).

Now, let $e = 2$. There are two cases. If $c(x) = 3$ and $k(x) = 1$, then $G$ must be $C_3$. Since $C_3$ must have $e = 2$, we need only consider the remaining case. Assume $c(x) = 2$ and $k(x) = 4$. This implies that $|G| = 8$. Since $c(x) < 8$, $G$ cannot be abelian. The only two groups of order 8 that are not abelian are $D_8$ and $Q_8$, the dihedral group of order 8 and quaternion group respectively. Neither of these groups have a self-centralizing involution and thus $G$ can be neither. This completes (3).
\end{proof}

\end{document}